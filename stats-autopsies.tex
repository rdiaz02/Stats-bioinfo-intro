\section[Help!]{When to call the statistician}

\begin{frame}
\frametitle{Tools}
\begin{itemize}
\item R and BioConductor: several packages.
\item Many, many, many (way toooooo many?) web-based tools. Some cited on
  first set of slides.
% \item Tnasas: \Burl{http://tnasas.bioinfo.cnio.es}
% \item CMA and paper: Slawski et al., \textit{BMC Bioinformatics}, 2008, 9: 439.
\end{itemize}
\end{frame}


\begin{frame}
  \frametitle{Statistical autopsies}
  \begin{quote}
    To call in the statistician after the experiment is done may be no more
    than asking him to perform a post-mortem examination: he may be able to
    say what the experiment died of.
    \vspace*{20pt}


    \textnormal{Sir Ronald Aylmer Fisher, Indian Statistical Congress, 1938}
  \end{quote}

\note{Hasta aquí, casi toda la charla parece una admonición}
\note{Sigamos con ese tono}
\note{Por lo del mejor prevenir que curar}

\end{frame}


\begin{frame}

\note{Pero el tono es mucho menos mal humorado}

  \frametitle{\ldots the alternative}
  \begin{quote}
    We want to foster the team concept, not the image of a statistical
    policeman arriving at the scene of a crime. Let's nip those false
    positives in the bud, not in the galleys.
    \vspace*{20pt}


    \textnormal{R.\ G.\ Easterling, \textit{The American Statistician}, 2010}
  \end{quote}
\end{frame}

%%% Local Variables: 
%%% mode: latex
%%% TeX-master: t
%%% End: 
